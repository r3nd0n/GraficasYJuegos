\documentclass[12pt]{article}
\usepackage{graphicx}
\usepackage{caption}
\usepackage{subcaption}
\usepackage{tikz}
\usepackage{venndiagram}
\usepackage{venndiagram}
\usepackage{tcolorbox}
\usepackage{listings}
\usepackage{enumitem}
\usepackage{amsmath}
\usepackage{amssymb}
\usepackage{colortbl}
\usepackage{xcolor}
\usepackage[margin=1cm, top=1.5cm, bottom=1.5cm]{geometry}

\tcbuselibrary{breakable}

\title{\textbf{Gráficas y Juegos: Tarea 02}}
\author{Martínez Méndez Ángel Antonio\\Pinzón Chan José Carlos\\Rendón Ávila Jesús Mateo}
\date{\today}

\begin{document}

\maketitle
\begin{center}
\vspace{3cm}
\includegraphics[width=0.195\textwidth]{Escudo.png}
\hspace{0.5cm}
\includegraphics[width=0.2\textwidth]{logo_ciencias.png}
\end{center}
\begin{center}
    \vspace{1cm}
    Universidad Nacional Autónoma de México\\
    Facultad de Ciencias\\
    Profesor: César Hernández Cruz\\
\end{center}

\newpage

%
% Ejercicio 1
%
\textbf{1.} Sea $G$ una gáfica, y recuerde que $c_G$ denota al número de componentes conexas de $G$.
Demuestre que si $e \in E$, entonces $C_G \leq c_{G - e} \leq c_G + 1$.\\

\begin{tcolorbox}[title=\textbf{Hipotesis}, colback=red!15!white, colframe=black!, breakable]

\end{tcolorbox}

\begin{tcolorbox}[title=\textbf{Definiciones}, colback=blue!15!white, colframe=black!, breakable]
    $Def$.
\end{tcolorbox}


\vspace{1cm}

%
% Ejercicio 2
%
\textbf{2.} Una gráfica es \textit{escindible} completa si su conjunto de vértices admite una partición $(S, K)$
de tal forma que S es un conjunto independiente, $K$ es un clan, y cada vértice en $S$ es
adyacente a cada vértice en $K$. Demuestre que una gráfica es escindible completa si y
sólo si no contiene a $C_4$ ni a $\overline{P_3}$ como subgráfica inducida. (Sugerencia: Un ejercicio de
la tarea anterior puede resultar de utilidad.)
\vspace{1cm}

%
% Ejercicio 3
%
\textbf{3}.

\begin{enumerate}[label=\alph*)]

    \item Demuestre que si $\mid E \mid > n - 1$, entonces $G$ es conexa.
    \begin{tcolorbox}[title=\textbf{Hipotesis}, colback=red!15!white, colframe=black!]

    \end{tcolorbox}
    \begin{tcolorbox}[title=\textbf{Definiciones}, colback=blue!15!white, colframe=black!]
    
    \end{tcolorbox}

    \item Para cada $n > 3$ encuentre una gráfica inconexa de orden $n$ con $|E| = n - 1$.
    \begin{tcolorbox}[title=\textbf{Hipotesis}, colback=red!15!white, colframe=black!]

    \end{tcolorbox}
    \begin{tcolorbox}[title=\textbf{Definiciones}, colback=blue!15!white, colframe=black!]
    
    \end{tcolorbox}
\end{enumerate}

\vspace{1cm}
%
% Ejercicio 4
%
\textbf{4.}


\begin{enumerate}[label=\alph*)]

    \item Demuestre que si $\delta > |V| - 1$, entonces $G$ es conexa.
    \begin{tcolorbox}[title=\textbf{Hipotesis}, colback=red!15!white, colframe=black!]

    \end{tcolorbox}
    \begin{tcolorbox}[title=\textbf{Definiciones}, colback=blue!15!white, colframe=black!]
    
    \end{tcolorbox}

    \item PAra $|V|$ par encuentre una gráfica $()$-regular e inconexa.
    \begin{tcolorbox}[title=\textbf{Hipotesis}, colback=red!15!white, colframe=black!]

    \end{tcolorbox}
    \begin{tcolorbox}[title=\textbf{Definiciones}, colback=blue!15!white, colframe=black!]
    
    \end{tcolorbox}
\end{enumerate}

\vspace{1cm}

%
% Ejercicio 5
%
\textbf{5.} Demuestre que si $D$ no tiene lazos y $\delta^+ \geq 1$, entonce s$D$ contiene un ciclo dirigido de longitus
al menos $\delta^+ + 1$.\\

\begin{tcolorbox}[title=\textbf{Definiciones}, colback=blue!15!white, colframe=black!]

\end{tcolorbox}


\end{document}