\documentclass[12pt]{article}
\usepackage{graphicx}
\usepackage{caption}
\usepackage{subcaption}
\usepackage{tikz}
\usepackage{tcolorbox}
\usepackage{listings}
\usepackage{amsmath}
\usepackage{amssymb}
\usepackage{xcolor}
\usepackage[margin=1cm, top=1.5cm, bottom=1.5cm]{geometry}

\tcbuselibrary{breakable}

\title{\textbf{Gráficas y Juegos: Tarea 02}}
\author{Martínez Méndez Ángel Antonio\\Pinzón Chan José Carlos\\Rendón Ávila Jesús Mateo}
\date{\today}

\begin{document}

\maketitle
\begin{center}
\vspace{3cm}
\includegraphics[width=0.195\textwidth]{Escudo.png}
\hspace{0.5cm}
\includegraphics[width=0.2\textwidth]{logo_ciencias.png}
\end{center}
\begin{center}
    \vspace{1cm}
    Universidad Nacional Autónoma de México\\
    Facultad de Ciencias\\
    Profesor: César Hernández Cruz\\
\end{center}

\newpage

%
% Ejercicio 1
%
\textbf{1.} Sea $D$ una digráfica de orden $n$. Demuestre que si $D$ no tiene ciclos dirigidos,
entonces existe un orden total, $v_1 , . . . , v_n$ de $V_D$ , tal que siempre que $(v_i , v_j)$ sea
una flecha de $D$, se tiene que $i < j$.

\begin{tcolorbox}[title=\textbf{Definiciones}, colback=blue!15!white, colframe=black!, breakable]
    $Def$. Un ciclo dirigido de tres o más vértices es una digráfica simple en la que su conjunto
    de vértices admite un orden cíclico de tal forma que dos vértices son adyacentes si y
    sólo si son consecutivos en el orden.
\end{tcolorbox}

\begin{tcolorbox}[title=\textbf{Hipotesis}, colback=red!15!white, colframe=black!, breakable]
    $D$ no tiene ciclos dirigidos
\end{tcolorbox}

Al trabajar en una digráfica acíclica, recordemos que existe por lo menos un vertice con ingrado cero, 
de lo contrario, D sería una digráfica con un ciclo dirigido inducido. Tomaremos a dicho vertice como el primero en nuestro orden 
de vertices debido a que, por definición, no hay ninguna flecha que incida en él. \\

Sea $D$ una digráfica simple, de orden $n$ y acíclica. Si el conjunto de sus vertices esta denotado 
por $V_D$  = \{$v_1, v_2  . . . , v_n$\}, i.e, tienen un orden $n$, podemos suponer que  $v_1 \in V_D $
 y demostramos por inducción que:\\

(Base) Si $D$ tiene solo un vertice, entonces $v_1 \in V_D$ tiene ingrado $d^-(v_1)$ = 0 y, necesariamente, cumple con ser un orden total.\\

(Hipotesis Inductiva) Supongamos que si $D$ tiene $k$ vertices, existe un orden $k$ de la forma 
$v_1, v_2, \dots, v_k$ tal que si $v_i v_j \in A_D$, por definición sabemos que debe existir $d^-(v_i)$ = 0, que si se cumple, 
podemos eliminar este vertice y la flecha que sale de él y comprobar para los siguientes dos vertices ($v_j v_j+1). 
En caso contrario, tendremos que probar esto mismo para los dos vertices anteriores $v_i-1 v_i.

(Paso inductivo) Sea $D$ una digrafica con $k + 1$ vertices.

Por definción, debe existir un vertice $v_i \in V_D$ con ingrado cero que cumpla con ser el primero en nuestro orden. Por H.I.
si vamos checando vertice por vertice $\{v_i, v_i+1, ..., v_k, v_k+1\}$ notaremos sin importar que si caemos en el caso en el que nuestro vertice tiene ingrado 1 o 0, llegaremos a nuestro caso base en v_k+1,
por lo que tendremos un orden de la forma v_k < v_i. 



\vspace{1cm}

%
% Ejercicio 2
%
\textbf{2.} Demuestre que si $G$ tiene diámetro mayor que 3, entonces $\overline{G}$ tiene diámetro menor que 3.

\begin{tcolorbox}[title=\textbf{Hipotesis}, colback=red!15!white, colframe=black!, breakable]
    $G$ es isomorfa a $H$.
\end{tcolorbox}
\begin{tcolorbox}[title=\textbf{Definiciones}, colback=blue!15!white, colframe=black!, breakable]
    $Def$. Una gráfica $G$ es \textbf{isomorfa} a una gráfica $H$ si existe una función $\varphi$ biyectiva, tal que:
    \[\varphi: V(G) \rightarrow V(H)\]
    Tal que para cualesquiera vertices $u,v \in V(G)$, se cumple que:
    \[uv \in E(G) \Leftrightarrow \varphi(u)\varphi(v) \in E(H)\]
    \\
    $Def$.El \textbf{complemento} de una gráfica $G$, denotada como $\overline{G}$, es la gráfica con el conjunto de vértices:
    \[V(\overline{G}) = V(G)\]
    y el conjunto de aristas:
    \[E(\overline{G}) = \{ uv \in V(G) \times V(G) \mid uv \notin E(G)\}\]
\end{tcolorbox}

\textbf{b)} Usando la definición de conexidad vista en clase, demuestre que si $G$ es inconexa,
entonces $\overline{G}$ es conexa.
\begin{tcolorbox}[title=\textbf{Hipotesis}, colback=red!15!white, colframe=black!]
    $G$ es inconexa, lo que implica la negación de conexidad.
\end{tcolorbox}
\begin{tcolorbox}[title=\textbf{Definiciones}, colback=blue!15!white, colframe=black!]
    $Def$. Una grafica $G$ es \textbf{conexa} si para cualquier partición $(X,Y)$ de $V(G)$, existe al menos una arista con un extremo
    en $X$ y el otro en $Y$.
\end{tcolorbox}

\vspace{1cm}

%
% Ejercicio 3
%
\textbf{3.} Sea $D$ una digráfica. Utilizando un análogo para digráficas de las matrices de incidencia, dmeuestra que:\\
\[
\sum_{\displaystyle v \in V_D} d^{+}(v) = \sum_{\displaystyle v \in V_D} d^{-}(v) = |A_D|
\]
\begin{tcolorbox}[title=\textbf{Hipotesis}, colback=red!15!white, colframe=black!]
    $D$ es una gŕafica dirigida (digráfica).
\end{tcolorbox}
\begin{tcolorbox}[title=\textbf{Definiciones}, colback=blue!15!white, colframe=black!]
    $Def$. Una gráfica $D$ es una \textbf{gráfica dirigida} si $D$ es una pareja ordenada $D=(V,A)$, donde
    $V$ es un conjunto arbitrario y $A$ es un subconjunto de $V \times V$ (parejas ordenadas).
    \\
    \\
    $Def$. La \textbf{matríz de incidencia} $M$ de $G$ es la matriz de $m \times n$ con entradas en $\{0, 1\}$ tal que:
    \[M_{ij} = \begin{cases} 1 & \text{si el vértice } v_i \in e_j\\ 0 & \text{en otro caso} \end{cases}\]

\end{tcolorbox}

$P.D.$
\[\sum_{\displaystyle v \in V_D} d^{+}(v) = \sum_{\displaystyle v \in V_D} d^{-}(v) = |A_D|\]
Sea $M^+$ y $M^-$ las matrices de incidencia de los exgrados e ingrados respectivamente, si tenemos una gráfica dirigida $D$ y su conjunto
de vértices $V(D)$ tal que:
\[V(D) = \{v_1, v_2, ..., v_n\}\]
y su conjunto de aristas $A(D)$ tal que:
\[A(D) = \{e_1, e_2, ..., e_m\}\]
Para la matríz de incidencia de una digráfica, tanto para los exgrados e ingrados, la suma
de los elementos de cada columna es igual a $1$. Sea la suma de los elementos de la columna igual a $0$,
se concluye que la arista no existe, por otro lado, si la suma es igual a $1$, la arista \textit{sale} de algún vértice (exgrado)
o se dirige a algún vértice (ingrado).\\

Suponiendo que la suma de la columna sea $\geq 2$, esto no es posible puesto que una arista sólo puede partir de un solo punto y viceversa, $i.e$ sólo 
puede dirigirse a un solo punto.\\

Como resultado al sumar todas las columnas de $M^+$ o $M^-$ obtenemos el número de aristas de la gráfica, $i.e$ se obtiene la 
cardinalidad de $A(D)$.

De forma similar, si sumamos las entradas del $i-esimo$ renglón de $M^+$ obtenemos el exgrado (o ingrado en el caso de $M^-$) del vértice $v_i$.\\

En base a lo anterior se puede argumentar que:
\[\mid A(D) \mid = \sum\limits_{j = 1}^{m} 1\]
\[= \sum\limits_{j = 1}^{m} \sum\limits_{n = 1}^{n} M^+_{ij}\]
\[= \sum\limits_{i = 1}^{n} \sum\limits_{j = 1}^{m} M^+_{ij}\]
\[= \sum\limits_{i = 1}^{n} d^+(v_i)\]
\[= \sum\limits_{i = 1}^{n} d^+(v_i)\]
\[= \sum_{\displaystyle v \in V(D)} d^+(v_i)\] 
Analogamente podemos hacerlo para ingrados, por lo que:
\[\mid A(D) \mid = \sum_{\displaystyle v \in V(D)} d^+(v) = \sum_{\displaystyle v \in V(D)} d^-(v)\]



\vspace{1cm}
%
% Ejercicio 4
%
\textbf{4.} Demuestre que cualesquiera dos trayectorias de longitud máxima en una gráfica
conexa tienen un vértice en común.\\

\begin{tcolorbox}[title=\textbf{Definiciones}, colback=blue!15!white, colframe=black!, breakable]
    $Def$. $G$ es conexa si para todos $u, v \in V$ existe un uv-camino.\\

    $Def$. Un trayectoria es un camino que no repite vertices.
\end{tcolorbox}

\begin{tcolorbox}[title=\textbf{Hipotesis}, colback=red!15!white, colframe=black!, breakable]
     Existen dos trayectorias de longitud máxima en una gráfica $G$ que es conexa. 
\end{tcolorbox}

Sea $G$ una gráfica conexa, en donde existen dos trayectorias P y Q de longitud máxima.\\

Supongamos que $P \cap Q = \emptyset$, i.e, que nuestras dos trayectorias de longitud máxima no 
comparten vertices en común. Como sabemos que $G$ es conexa, si tomamos cualquier par de vértices de $P$ y $Q$, 
digamos $u \in P$ y $v \in Q$, entonces existe un uv-camino en $G$ que los conecta.\\

Imaginemos que $F$ es ese uv-camino que conecta a los vertices que tomamos de forma arbitrariamente 
de P y Q. Ahora, si consideremos la trayectoria generada al tomar un vertice de $P$ desde uno de sus extremos, digamos $v_i$ hasta $u$, seguida por el camino dado
$F$, y luego la parte de $Q$ desde un $u_i$ hasta uno de sus extremos. Esta nueva trayectoria es estrictamente 
más larga que $P$ y $Q$, lo que contradice la suposición de que $P$ y $Q$ eran de longitud máxima.\\

Lo que nos arroja una contradicción, por lo que esto implica que cualquier par de trayectorias de longitud máxima en cualquier gráfica $G$ deben compartir al menos un vértice.

%
% Ejercicio 5
%
\textbf{5.} 

%
% Ejercicio 6 
%
\textbf{6.} Demuestre que si $\mid E \mid \geq\mid V \mid$, entonces G contiene un ciclo.

\begin{tcolorbox}[title=\textbf{Hipotesis}, colback=red!15!white, colframe=black!, breakable]
    $\mid E \mid \geq \mid V \mid$.
\end{tcolorbox}

\begin{tcolorbox}[title=\textbf{Definiciones}, colback=blue!15!white, colframe=black!]
    $Def$. Un \textbf{ciclo} de tres o mas vértices es una gráfica simple en la que su conjunto de
    vértices admite un orden cíclico de tal forma que dos vértices son adyacentes si y sólo si son consecutivos en el orden.\\

    $Def$. Una \textbf{trayectoria} es una gráfica simple en la que su conjunto de vértices admite un orfen lineal de tal dforma 
    que dos vértices son adyacentes si y sólo si son consecutivos en el orden.
\end{tcolorbox}

$P.D$ Existe alguna arista $v_0 v_n \in E(G)$ tal que $v_n v_0 \in E(G)$ para algún orden de vértices cíclico, digamos $M$, con $M \subseteq V(G)$.\\

Sea un orden lineal $(m_0, m_1, m_2, \dots , m_n)$ de vértices de $G$, donde $n = \mid V \mid$, tenemos un atrayectoria si $m_i$ es adyacente a $m_i+1$
para $i \in \{0, 1, 2, 3, \dots, n\}$.\\

Como se cumplira siempre, por la condición, la existencia de las aristas $m_0 m_1, m_1 m_2, m_2 m_3, \dots , m_{n-1} m_n$ tendremos que 
$\mid E \mid < \mid V \mid$ pues existen dos vértices en la secuencia que no serán adyacentes entre sí, en este caso $m_0$ y $m_n$. Con lo anterior 
tendríamos $\mid E \mid = n - 1$.\\

Con ello, si añadimos un arista $m_n m_i$ al conjunto $E(G)$ tal que $i = \{0, 1, 2, \dots, n-2\}$ (usamos $n-2$ pues si permitimos $i = n-1$ esa arista ya existe y no nos sirve).
Como ya hemos garantizado una trayectoria, cualquier arista $m_n m_i$ que añadamos a $E(G)$ nos garantiza un ciclo en $G$ cuando $\mid E \mid \geq \mid V \mid$.






\end{document}